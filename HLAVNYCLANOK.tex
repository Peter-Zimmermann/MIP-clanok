% Metódy inžinierskej práce

\documentclass[10pt,twoside,slovak,a4paper]{article}

\usepackage[slovak]{babel}
%\usepackage[T1]{fontenc}
\usepackage[IL2]{fontenc} % lepšia sadzba písmena Ľ než v T1
\usepackage[utf8]{inputenc}
\usepackage{graphicx}
\usepackage{url} % príkaz \url na formátovanie URL
\usepackage{hyperref} % odkazy v texte budú aktívne (pri niektorých triedach dokumentov spôsobuje posun textu)

\usepackage{cite}
%\usepackage{times}

\pagestyle{headings}

\title{Softvér v automobilovom priemysle\thanks{Semestrálny projekt v predmete Metódy inžinierskej práce, ak. rok 2021/22, vedenie: Ing. Fedor Lehocki, Phd.}} % meno a priezvisko vyučujúceho na cvičeniach

\author{Peter Zimmermann\\[2pt]
	{\small Slovenská technická univerzita v Bratislave}\\
	{\small Fakulta informatiky a informačných technológií}\\
	{\small \texttt{xzimmermann@stuba.sk}}
	}

\date{\small 11. október 2021} % upravte



\begin{document}

\maketitle

\begin{abstract}
\ldots Automobilový priemysel stle napreduje a každým rokom sa dáva čoraz väčši dôraz na softvér, ako na samotné mechanické vlastnosti alebo diely vozidiel. Pri vývoji automobilových softvérov je kľúčové využívať vopred stanovené postupy, ktoré pozostávajú z budovania systému pre daný softvér, jeho samotný vývoj a následná prevádzka. Realizáciu týchto postupov zabezpečujú modely, s ktorými je realizácia intenzívnejšia, keďže dopredu poznáme všetky fázy (V-Model), alebo model, ktorý nemá vopred stanovené pevné fázy, takže je viac otvorený a kreatívnejší (Agile Model). Jednou z najdôležitejších činností je testovanie už vyvinutého softvéru, ktorým treba zistiť, či je prevádzka daného softvéru plne bezpečná a spoľahlivá. Taktiež je v práci ukázane, ako odkáže moderný softvér v automobiloch predpokladať možnu škodu pri vzniknutom nebezpečenstve a zabrániť jej.
\end{abstract}



\section{Úvod}

Motivujte čitateľa a vysvetlite, o čom píšete. Úvod sa väčšinou nedelí na časti.

Uveďte explicitne štruktúru článku. Tu je nejaký príklad.
Základný problém, ktorý bol naznačený v úvode, je podrobnejšie vysvetlený v časti~\ref{nejaka}.
Dôležité súvislosti sú uvedené v častiach~\ref{dolezita} a~\ref{dolezitejsia}.
Záverečné poznámky prináša časť~\ref{zaver}.



\section{Postupy vo vývoji automobilového softvéru} \label{Postupy}
V tejto sekcii sa primárne zameriavam na tri hlavné postupy pri vývoji automobilového softvéru.

Vývoj automobilového softvéru a jeho samotné aplikovanie je v dnešnej dobe jeden z najhlavnejších procesov pri zhotovovaní automobilu. Je to proces, ktorý pozostáva z viacerých postupov: 
\begin{enumerate}
\item Budovanie systému
\item Vývoj softvéru
\item Prevádzka
\end{enumerate}

Prvý z postupov procesu je budovanie systému. V tomto postupe sa výrobcovia originálnych zariadení obvykle zaoberajú požiadavkami, ktoré sú primárne zamerané na funkčnosť zariadenia, ktoré bude najviac vyhovovať stanoveným požiadavkám (npr. riadiaca jednotka, hardvér palubného počítača). Títo výrobcovia sa vo väčšine prípadov riadia architektúrou automobilového open systému s názvom „AUTOSAR“. AUTOSAR je v dnešnej dobe celosvetové vývojove partnerstvo v oblasti softvéru a elektroniky v automobiloch. 

Druhým z postupov je vývoj samotného softvéru, ktorého hlavným cieľom je vytvoriť viacero programovacích kódov, z ktorých sa vyberajú tie najvhodnejšie, ktoré budú najlepšie pracovať s už zvolenými zariadeniami. Aby sa dokázali rozlíšiť rôzne úrovne zložitosti softvéru, možno použiť dva najbežnejšie modely a V-Model a Agile model.  
Posledným postupom je samotná prevádzka. Znamená to, že každý softvér je aj po vyrobení a uvedenia do predaju udržiavaný aktualizáciami, aby prevádzka a funkčnosť samotného softvéru bola čo najdlhšia. \cite{Coplien:MPD}




\section{Modely pre vývoj automobilového softvéru} \label{modely}

Táto sekcia slúži na priblíženie dvoch široko známych modelov vo vývoji softvérov pre automobily.

\subsection{V-Model} \label{vmodel}

V-Model je v dnešnej dobe najpoužívanejší model vo automobilovom priemysle. Je rozdelený do troch častí. Prvou časťou je definovanie si krokov. Po tejto časti prichádza na rad samotné programovanie, po ktorom nasleduje overovanie predošlých krokov. Funguje to na tom princípe, že sa po fáze programovania otočí naspäť nahor na overovanie a výsledok každej fázy sa overí ešte predtým, ako sa prejde do nasledujúcej fázy V-Modelu.

\subsection{Agile Model} \label{agilemodel}

Tento model z veľkej časti dáva do popredia evolučnný rozvoj a rapídne zmeny. Vo viacerých prijektoch sa softvéroví inžinieri rozhodnú pre tento spôsob z dôvodu, že vďaka nemu dokážu spracovať aj nepredpokladané zmeny zo strany zákaznických požiadaviek. Medzi najväčšie výhody patrí to, že nemá pevné fázy, na rozdiel od V-Modelu. Z tohto dôvodu je Agile Model viac flexibilný, čo má za následok zvýšenie produktivity. 

\subsection{Porovnávanie V-Modelu a Agile Modelu} \label{porovnanie}

Obidva modely su známe najmä z toho dôvodu, že sú jednoduché. Agile Model je navrhnutý tak, aby bol čo najmenej zložitý vo vývojových procesoch. Ale z dôvodu, že softvér v automobiloch vyžaduje správnu spojitosť meddzi softvérovou intergritou, sa v praxi častejšie využíva V-Model. Okrem toho, Agile Model ma viacero nevýhod, ktoré sa vo vývoji softvéru pre automobily považujú za kľúčové , ako napríklad: slabá softvérová integrita, obmedzené znovu použitie kódu, nepriaznivé efekty na kvalitu práce a zložitá organizácia celej štruktúry. 



\subsection{Ešte nejaké vysvetlenie} \label{ina:este}

\paragraph{Veľmi dôležitá poznámka.}
Niekedy je potrebné nadpisom označiť odsek. Text pokračuje hneď za nadpisom.



\section{Testovanie} \label{testovanie}

\subsection{Komplexné informácie k testovaniu softvéru} \label{komplexnetestovanie}



\section{Porovnávanie dvoch popredných automobiliek z hľadiska softvéru} \label{automobilky}




\section{Záver} \label{zaver} % prípadne iný variant názvu



%\acknowledgement{Ak niekomu chcete poďakovať\ldots}


% týmto sa generuje zoznam literatúry z obsahu súboru .bib podľa toho, na čo sa v článku odkazujete
\bibliography{HLAVNALITERATURA}
\bibliographystyle{plain} % prípadne alpha, abbrv alebo hociktorý iný
\end{document}
